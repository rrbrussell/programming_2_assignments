\documentclass[letterpaper,oneside]{scrartcl}
\usepackage[utf8x]{inputenc}
\usepackage{amsmath}
\usepackage{fancyvrb}

\title{Programming 2\\Assignment Number 4}
\author{Robert R. Russell}
\date{\today}

\begin{document}

\maketitle

\section{Definition Of Problem}
This assignment is to create a program that could fill an array with up to 100000 random numbers.
The choice of numbers is given on the command line.
\section{Preliminary Analysis}
The basic steps of the problem is:
\begin{enumerate}
\item Find out how many numbers to generate and sort.
  \begin{enumerate}
  \item Verify the number of command line arguments.
  \item Convert the numbers argument to a number.
  \item Verify it is within the bounds of the problem.
  \end{enumerate}
\item Fill array with random numbers.
  \begin{enumerate}
  \item Seed the random number generator.
  \item Generate the random numbers storing them in the array.
  \end{enumerate}
\item Sort the numbers.
\end{enumerate}

\section{Code and Output}
\fvset{numbers=left}

Note code protected by
\begin{verbatim}
#ifdef PRINTRANDOM
\end{verbatim}
is debuging code.

\subsection{main.c}
\VerbatimInput{main.c}

\subsection{random.h}
\VerbatimInput{random.h}

\subsection{random.c}
\VerbatimInput{random.c}

\subsection{sort.h}
\VerbatimInput{sort.h}

\subsection{sort.c}
\VerbatimInput{sort.c}

\subsection{printrandoms.h}
\VerbatimInput{printrandoms.h}

\subsection{printrandoms.c}
\VerbatimInput{printrandoms.c}

\section{Final Analysis}
This program was a fairly simple program to write assuming you use a simple
sorter, such as a bubble or selection sort. Not a heap sort which is what I
tried to use first.

\end{document}
