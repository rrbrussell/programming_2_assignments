\documentclass[letterpaper,oneside]{scrartcl}
\usepackage[utf8x]{inputenc}
\usepackage{amsmath}
\usepackage{fancyvrb}

\title{Programming 2\\Assignment Number 2}
\author{Robert R. Russell}
\date{January 28, 2001}

\begin{document}

\maketitle

\section{Definition Of Problem}
We are to write a program that calculates a table of square roots for
a given range of numbers with a given increment. The range and increment will be
given as command line arguments.

\section{Preliminary Analysis}
The primary problems with this code will be determining whether the start and
end of a range are valid with a given increment. There are only two possible
valid combinations.
\begin{enumerate}
\item start $<$ end and increment $>$ 0
\item start $>$ end and increment $<$ 0
\end{enumerate}

\newpage 
\section{Code and Output}
\fvset{numbers=left}

\subsection{Makefile}
\VerbatimInput{Makefile}

\subsection{common\_datatypes.h}
\VerbatimInput{common_datatypes.h}

\subsection{constraints.h}
\VerbatimInput{constraints.h}

\newpage
\subsection{main.c}
\VerbatimInput{main.c}

\subsection{table\_generator.h}
\VerbatimInput{table_generator.h}

\subsection{table\_generator.c}
\VerbatimInput{table_generator.c}

\subsection{Runs}
\begin{verbatim}./table -1 23 3\end{verbatim}
\begin{tabular}{l|l}
Number & Square Root \\ \hline
-1 & 1.0000$\imath$ \\ 
2 & 1.4142 \\ 
5 & 2.2361 \\ 
8 & 2.8284 \\ 
11 & 3.3166 \\ 
14 & 3.7417 \\ 
17 & 4.1231 \\ 
20 & 4.4721 \\ 
23 & 4.7958 \\ 
\end{tabular}

\begin{verbatim}./table 23 4 -3\end{verbatim}
\begin{tabular}{l|l}
Number & Square Root \\ \hline
23 & 4.7958 \\ 
20 & 4.4721 \\ 
17 & 4.1231 \\ 
14 & 3.7417 \\ 
11 & 3.3166 \\ 
8 & 2.8284 \\ 
5 & 2.2361 \\ 
\end{tabular}


\section{Final Analysis}
This assignment was more difficult than the previous assignment for two reasons.
primarily because of the complexity of the program. The program we were assigned
to write has several major conditions that must be met:
\begin{enumerate}
\item Enforce the correct number and parsing of the command line arguments.
\item Variable direction and length of iteration. The loop conditional changes
depends on the direction of iteration.
\item Correct handling of negative values. The C sqrt() is only defined for
positive real numbers. It will not work with negative numbers.
\item Correct handling of the starting and ending points of the range if the 
direction of iteration is not correct. 
\end{enumerate}
While these conditions are not hard to satisfy, doing so while keeping a maintable
source tree does require some up front thinking. Even after Software Design and
Development I still have a habit of coding before thinking.

\end{document}
