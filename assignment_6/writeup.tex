\documentclass[letterpaper,oneside]{scrartcl}
\usepackage[utf8x]{inputenc}
\usepackage{amsmath}
\usepackage{fancyvrb}

\title{Programming 2\\Assignment Number 6}
\author{Robert R. Russell}
\date{April 15, 2011}

\begin{document}

\maketitle

\section{Definition Of Problem}
This assignment is to create a queue of integers borrowing the list from our
previous assignment.

The stack must satisfy the following criteria:
\begin{enumerate}
\item Provide for pushing to a stack.
\item Provide for popping from a stack.
\item Provide a check for the empty condition.
\item Provide a check for the full condition.
\item Provide a Type-def for the data structure used to implement the stack.
\item Provide an initialization method to setup a stack.
\end{enumerate}

\section{Preliminary Analysis}

This basic implementation uses a linked list to implement the data portion of the implementation.
A linked list consists of a series of nodes and links describing the ordering of the nodes.
A node in a linked list is a structure that contains a data item and a pointer to the next node in the list.

The functions are also provided in with the data structure.

The menu for the user is in main.

\section{Code and Output}
\fvset{numbers=left}

\subsection{main.c}
\VerbatimInput{main.c}

\subsection{queue.h}
\VerbatimInput{queue.h}

\subsection{queue.c}
\VerbatimInput{queue.c}

\subsection{Output}
I could not get the program to work successfully because the ``theStack'' variable in main
would not update correctly upon the return from pushStack;

\begin{Verbatim}
Welcome to stacker
Main Menu
1. Add to stack
2. Delete from stack
3. List the stack
4. Quit
Your Choice: 1
Enter a number: 342
New node allocated at 0x1ef0070
theStack is now (nil)
Main Menu
1. Add to stack
2. Delete from stack
3. List the stack
4. Quit
Your Choice: 4
bye
\end{Verbatim}

\section{Final Analysis}

A simple program great a exploring C's memory handling capabilities.

\end{document}
